\documentclass[final,1p,authoryear]{elsarticle}
%% for a journal layout:
%% \documentclass[final,1p,times]{elsarticle}
%% \documentclass[final,1p,times,twocolumn]{elsarticle}
%% \documentclass[final,3p,times]{elsarticle}
%% \documentclass[final,3p,times,twocolumn]{elsarticle}
%% \documentclass[final,5p,times]{elsarticle}
%% \documentclass[final,5p,times,twocolumn]{elsarticle}


\date{October, 2020}

\usepackage{hyperref}




\begin{document}


\begin{frontmatter}


\title{Corporate Governance: A Brief Review of the Literature\tnoteref{t1,t2}}

\tnotetext[t1]{
	In this article, the corporate governance literature is concisely reviewed, but it is worth mentioning that it is not a deep dive into the subject, and also several other corporate governance mechanisms are not covered.}
	
\tnotetext[t2]{
	The article extensively borrows from the papers in the references section.}
	

\author[1]{Mahdi Mir (Author)}
\ead{MrMahdiMir@gmail.com}

\author[1]{Mahdi Heidari (Supervisor)}
\ead{m.heidari@teias.institute}
\ead[url]{teias.institute/people/faculty/econ/mahdi-heidari}

\address[1]{Tehran Institute for Advanced Studies (TeIAS), 17 East Daneshvar St.,
North Shirazi St., Mollasadra Blvd., Tehran, Iran.}

%% use the tnoteref command within \title for footnotes;
%% use the tnotetext command for the associated footnote;
%% use the fnref command within \author or \address for footnotes;
%% use the fntext command for the associated footnote;
%% use the corref command within \author for corresponding author footnotes;
%% use the cortext command for the associated footnote;
%% use the ead command for the email address,
%% and the form \ead[url] for the home page:
%%
%% \title{Title\tnoteref{label1}}
%% \tnotetext[label1]{}
%% \author{Name\corref{cor1}\fnref{label2}}
%% \ead{email address}
%% \ead[url]{home page}
%% \fntext[label2]{}
%% \cortext[cor1]{}
%% \address{Address\fnref{label3}}
%% \fntext[label3]{}

%% use optional labels to link authors explicitly to addresses:
%% \author[label1,label2]{<author name>}
%% \address[label1]{<address>}
%% \address[label2]{<address>}


\begin{abstract}

	At first, this article briefly describes the subject of corporate governance, why it is essential, and why product market competition is not enough for it. Moreover, the first section discusses the main agency problem that corporate governance deals with, and it argues about how financing can occur without governance. The two following sections concisely review two major approaches to corporate governance; first, the legal protection of investors, i.e., shareholders and creditors; and second, the effects of large investors' presence. They analyze these two approaches, both in theory and empirical studies. The final section mentions the critical role of financial analysts as an instance of other corporate governance mechanisms.
	
\end{abstract}



\begin{keyword}
Corporate Governance \sep Ownership Structure \sep Investor Legal Protection \sep Large Investors
\end{keyword}



\end{frontmatter}



\section{Introduction}

The subject of corporate governance is of enormous practical importance. Corporate governance deals with mechanisms through which investors assure themselves of getting a return on their corporations' investments. The question is, how do capital suppliers get managers to return some of the profits to them? How investors make sure that managers do not steal their funds or invest in bad projects? How do they monitor and influence managers?

In perspective, corporate governance literature deals with the principal-agent problem that stems from the separation of ownership and control or the separation of firms' financing and management in theory and practice. Here, the agency problem refers to difficulties that financiers have in ascertaining that their funds are not expropriated or wasted in destructive projects. \citet[P.4]{doi:10.1111/1468-0262.00177} defines corporate governance as "the design of institutions that induce management to internalize the welfare of stakeholders." It means that not just the welfare of shareholders and creditors matters in corporate governance but also stakeholders, such as employees, customers, suppliers, and communities.

Many studies have shown, a better corporate governance environment is associated with a more developed financial system. Moreover, a more developed financial sector facilitates economic growth. In fact, in an influential study, \cite{10.2307/116849} show that financial development has a causal effect on growth; specifically, it facilitates growth. They argue that financial development reduces external finance costs to firms, which is more crucial for technologically more dependent industries on external finance. Therefore, the effect of corporate governance on economic growth through financial development raises its importance. 

Most advanced market economies have solved the corporate governance problem reasonably well, in that they have assured the flows of enormous capital to firms and actual return of some of the profits to finance providers. Nevertheless, this does not mean it has been solved entirely, and corporate governance measures cannot be improved. The United States, Germany, Japan, and the United Kingdom have some of the world's best corporate governance environments. These countries' corporate governance systems are mostly different from other countries relative to their differences. The Italian corporate governance system is so underdeveloped that virtually retard external finance flow to firms. In less developed economies, corporate governance mechanisms are practically nonexistent. In Russia, the lack of effective corporate governance mechanisms leads to substantial diversion of assets by managers of many privatized firms and the virtual nonexistent external finance to firms. Understanding corporate governance sheds light on the discussion of marginal improvements of prosperous economies and stimulating major institutional reforms where they need to be made.

One might think that we should not worry about corporate governance because, in the long run, the product market competition would take care of corporate governance. After all, competition forces firms to minimize costs so they would adopt corporate governance mechanisms to raise the external capital at the lowest cost. However, market competition is probably the most potent force toward economic efficiency; it is doubtful to be enough for corporate governance. Market competition reduces the return on capital and reduces the amount that managers can expropriate, but it does not prevent managers from expropriating competitive return after the capital is sunk. Hence solving this problem requires something more than the competition. This evolutionary view of economic change will be discussed in more detail later in the following.

The rest of the paper is organized as follows. Section 2 describes the agency problem and how finance can occur without governance. Section 3 discusses investor legal protection. Section 4 argues about the effects of large investors. Section 5 briefly describes the role of financial analysts as an instance of other corporate governance measures.


\section{The Agency Problem}
\subsection{Contracts and Management Discretion}

The essence of the corporate governance agency problem is the separation of ownership and control. The financiers and the manager sign a contract. Ideally, this contract would specify what the manager does in all states of the world and how the returns are divided. However, this is technologically infeasible; hence the manager and financiers have to allocate residual rights, i.e., the decision rights in the circumstances not foreseen by the contract.

Because financiers do not have the information and qualification to decide in these residual states, the very reason they hire managers in the first place, they cannot retain all residual rights; consequently, managers end up with substantial residual rights and, hence, discretion to allocate funds and profits.

Therefore, those who control a firm, either managers or controlling shareholders (hereafter insiders), can divert corporate wealth to themselves instead of sharing it with other investors. In other words, insiders can expropriate outside shareholders or creditors.

Expropriation can take a variety of forms. In the most straightforward instance, it can take the form of outright theft of corporate assets; in another instance, insiders sell outputs, assets, or other securities to other firms they own below market prices. Excessive compensation to management, diversion of corporate opportunities, installing possibly unqualified family members in managerial positions, directed equity issuance, or personal loans to insiders are some other forms of expropriation. Formally, insiders' ability to divert corporate wealth to themselves is called "private benefits of control." Much of corporate governance deals with limits managers put on themselves or investors put on managers to limit expropriation to reduce the ex-post misallocation and induce investors to provide more funds to firms ex-ante.

\subsection{Pieces of Evidence on Agency Costs}

Empirically, the private benefits of control have been investigated in several contexts. A considerable amount of research has documented the prevalence of this self-serving managerial behavior. Most of these studies are event studies based on the idea that if the stock price falls after an announcement of managers' actions, this must serve managers' interests rather than shareholders' interests. However, this inference is not justified in some circumstances, but such event studies are reasonably compelling. Such event studies have become the most common empirical methodology in corporate governance and finance.

The term "tunneling" refers to expropriating minority shareholders with the diversion of a firm's resources to its controlling shareholder (who is typically also on the management board). Such a controlling shareholder is common in Western and Eastern Europe, Latin America, and Asia (\cite{doi:10.1111/0022-1082.00115}). Owners of business groups are often accused of tunneling resources from the firms with low cash flow rights to firms where they have high cash flow rights. Typically, in a business group, a single controlling shareholder (or a family) controls several independently traded firms, yet he has a significant cash flow rights in only a few of them. These differences in cash flow rights across different firms in a group create a strong incentive to expropriate. \cite{RePEc:oup:qjecon:v:117:y:2002:i:1:p:121-148.} propose a general methodology to identify and measure the extent of tunneling. Their methodology rests on isolating earning shocks and testing the distinctive implications of tunneling hypothesis about these shocks' propagation across group firms. They apply this methodology to Indian business groups and find a large amount of tunneling. They also find that market prices partly incorporate tunneling, meaning that the stock market penalizes tunneling activities at least to some extent.
 
Also, \cite{doi:10.1111/1540-6261.00510} find that substantial tunneling occurs in Korean business groups (Chaebols). They find that on average, when a business group firm makes an acquisition, its stock price falls, and other firms' value in the group increases, meaning controlling shareholder tunnels firm resources to increase personal wealth. The concentrated ownership by owner-managers in chaebol bidders and rescue mergers by chaebol bidders is negatively related to bidder announcement returns but is positively related to the value-weighted announcement returns portfolio of other firms in the same group. A chaebol bidder loses its value, but other firms in the same chaebol rise in value suggest a wealth transfer from the bidding firm to the other firms in the same group. They also find that the mean market value change of insider holdings in chaebol bidders around the time of the merger announcement is -105 million won, but that in other member firms, it is 614 million won. All these findings support the tunneling hypothesis in business groups. In China, \cite{JIANG20101} find that during 1996–2006, controlling shareholders (insiders) borrowed tens of billions of RMB (the Chinese currency), usually interest-free and rarely paid back, from hundreds of Chinese listed companies. They also find a higher implied discount rate in the valuations of firms' earnings with large amounts of intercorporate loans.

Moreover, insiders have a strong incentive to hold excessive amounts of cash for their benefits, as \cite{10.2307/1818789} free cash flow hypothesis suggests for managers. Cash provides more expropriation opportunities to insiders because it can be conveniently transformed into personal benefits. As an indirect piece of evidence, in China, \cite{chenetal2012} find that firms hold extensively lower amounts of cash after the split share structure reform, suggesting their suboptimal behavior before the reform. The split share structure reform took place in China in 2005 that allowed previously non-tradable shares held by controlling shareholders to be freely tradeable on the exchanges. This reform aligned the incentives of controlling shareholders more in line with those of minority shareholders. The reform by removing significant market friction caused a shock to the corporate governance system and increased the incentive for large shareholders to be concerned about share prices.

Consequently, the reform reduced insiders' incentives to expropriate firm resources because they could realize gains and obtain cash by selling shares; before the reform, they could obtain cash only from cash distributions, including, possibly, tunneling activities or related party transactions. Therefore, these findings are consistent with the premise that in a better corporate governance environment controlling shareholders expropriate less, leading to a significant decrease in firms' cash holdings. Significantly, this effect is more pronounced in firms in which governance arrangements were weakest before the reform.


\subsection{Financing Without Governance}

The previous arguments raise the central question of corporate governance that is why financiers invest when they are aware of managers' vast discretion in allocating funds and other firm's decisions, often to the point of being able to expropriate much of it. A few following paragraphs describe two explanations to external finance puzzle that do not rely on corporate proper: first, the idea of managers' reputation, and second, the idea that investors are gullible and get taken. Both of these approaches have the common feature of investors not getting any rights, hoping that they would make money in the future.

The reputation building argument is that managers payback investors because they want to raise external finance in the future in the capital market; and in order to convince future investors, they have to build a reputation. However, pure reputational stories run into a backward recursion problem. At a point in time, for the manager, when the future benefits of raising funds are less than the costs of paying promised payments, the manager rationally default. Moreover, because investors know that such a time will be reached in the future, they would not finance the firm in the first place. Hence there is no possibility of external finance. An alternative explanation is that investors get excited about companies and finance them without thinking much about getting their money back only because of excessive optimism and short-term share price appreciation. As the evidence suggests, managers can expropriate only limited wealth, and the securities do have some underlying value, so the explanations for this have to go beyond investors' optimism.


\section{Investor Legal Protection}

The principal reason that financiers invest in firms is that they receive control rights in exchange. The most important rights investors have; are voting on most important corporate issues like board elections, firm liquidation, or acquisitions. When investor rights are extensive and well-enforced by regulators, they are willing to finance firms; in contrast, when the legal system does not protect outsiders, the external finance would break down. All outsiders, large or small, shareholders or creditors need to have their rights protected. Much evidence suggests that the level of investor protection provided by the legal system has significant economic consequences. Much of the difference in corporate governance systems worldwide stem from managers' legal obligations to financiers and how courts interpret and enforce these laws. Variations in law and its enforcement quality are crucial to understanding why firms in some countries raise significantly more external funds than the others.

Moreover, traditional comparisons of corporate governance systems focus on the institutions financing firms rather than on the legal protection of investors: precisely, the distinction between bank-centered corporate governance systems like Germany and Japan, and market-centered corporate governance systems such as those of the United States and U.K. \cite{19455} argue this approach is not a useful way to distinguish financial systems, and legal approach is a more fruitful way to understanding corporate governance and its reform. To better explaining the critical role of the legal system in corporate governance and its effect on the size and breadth of the debt and equity markets; the next subsection presents a simple model introduced by \cite{19465}, after that some empirical evidence about the role of the legal system will be explained. Then the Coasian view in corporate governance is examined. Finally, at the end of the section, some unanswered questions are introduced.

\subsection{A Simple Model of Investor Legal Protection}

\cite{19465} introduce a simple two date model of an entrepreneur going public in an environment with limited shareholder protection. They examine entrepreneur decisions and market equilibrium. The implications of this simple model are consistent with several empirical studies. Assume an entrepreneur is trying to raise external finance for a project and deciding how much equity to sell and how big the project to undertake. The entrepreneur operates in an environment with limited investor protection so he can divert only some of the profits after they materialize. By doing so, he risks being sued and fined. The investor protection environment's quality in this model is given by the probability that the entrepreneur is caught for expropriating from shareholders. They assume that entrepreneurs retain control of their firms regardless of the fraction of the cash flow rights they sell. The market interest rate is determined by supply and demand for funds. The entrepreneur decides to divert a fraction of the realized profits. If he is caught, he is forced to return the diverted amount and pay a fine, and the entire revenue is distributed as dividends. If the entrepreneur is missed, he keeps the diverted amount, and the remaining is distributed as dividends. In the model, countries differ in the level of legal protection they provide to investors. that is, the probability of the entrepreneur gets caught.

With the premise of perfect capital mobility between countries, the model implies that for two countries that differ in the level of investor protection; the country with higher investor protection level, that is more likely the entrepreneur gets caught, has lower ownership concentration, broader external capital markets, and larger firms. Also, it shows that more firms go public in countries with better investor protection. Moreover, equilibrium diversion decreases as the level of investor protection rises. These implications are consistent with the evidence. With the assumption of no capital mobility, each country has its equilibrium interest rate; in this situation, the model implies that the country with better investor protection has a higher market interest rate. Compared to the results derived for perfect capital mobility, capital markets with no mobility are again more extensive, and there is also more investment in countries with better investor protection, but the difference is smaller due to the effect of a higher interest rate. 

In the case of no capital mobility, in countries with better investor protection, not only more funds are raised by firms, but these funds are also channeled to higher-productivity projects. This result holds since better investor protection leads high-productivity firms to demand more funds. The increased demand raises the country's interest rate. As a result, in countries with adequate investor protection, entrepreneurs with moderately productive projects supply their funds to the more productive firms in the market instead of undertaking their own less productive projects. This result sheds light on an important question raised by \cite{RePEc:aea:aecrev:v:80:y:1990:i:2:p:92-96}: in less developed economies, a standard production function implies greater marginal product of capital and hence higher interest rate than rich countries, so why does not capital flow from developed to developing countries? This effect gives another explanation that the real interest is higher in better investor protection countries, eliminating the incentive for capital to flow to worse investor protection countries because too much of it is expropriated in the worse investor protection country. This argument raises the question: why do not countries with bad investor protection systems suffering from financial underdevelopment improve their legal environment?


\subsection{Evidence on Investor Legal Protection}

Strong law and well-developed institutions can help resolve agency problem and protect minority shareholders. Research suggests that the level of investor legal protection is an essential determinant of financial market development. In an environment with laws that protect external financers and well-enforced, outside investors are willing to supply capital to firms, and financial markets are broader. In contrast, where laws do not protect outsiders, investors are unwilling to finance firms, and the development of financial markets is stunted. In this subsection, a few empirical studies are reviewed, investigating the role of investor legal protection in corporate governance and its consequent economic outcomes.

In a critical study by \cite{10.2307/2329518}, using the data of 49 countries, they find that countries with better investor protection, measured by both the character of the legal rules and the quality of law enforcement, have a larger and broader capital market, both of equity and debt markets. They also find a systematic difference between countries from different legal origins; specifically, French civil law originated countries have both the weakest investor protections and least developed financial markets, especially compared to English common law origin countries. Furthermore, \cite{19501} present a new measure of investor protection against expropriation by insiders across countries, namely the anti-self-dealing index. They show that a high anti-self-dealing index is associated with valuable stock markets, more domestic firms, more IPOs, and lower private benefits of control. The index is also a statistically significant and economically strong predictor of various stock market development measures across countries, and it is related to a more developed securities market. In another paper, \cite{doi:10.1111/1540-6261.00457}, by developing a simple model and testing with 27 countries' data, show that consistent with the model, firms in better investor protection countries have higher valuation. 

In an interesting research, \cite{JOHNSON2000141} find that in the Asian financial crisis 1997-98, which affected all the emerging markets open to capital flows, measures of corporate governance, particularly the level of protection for minority shareholders, explain the extent of exchange rate depreciation and stock market decline better than do macroeconomics measures. They argue that as long as growth lasts, the institutions that protect shareholders and creditors' rights are unnecessary, but when growth prospects decline, the lack of good corporate governance becomes crucial. Because expropriation by managers increases when the expected rate of return on investment falls, then an adverse shock to investor confidence will lead to increased expropriation as well as lower capital inflow and more significant attempted capital outflow for a country leading to exchange rate depreciation and stock market decline.

Also, Poor investor protection can harm share liquidity. \cite{doi:10.1111/1540-6261.00551} compare Hong Kong-based blue-chip firms and China-based firms listed on the Hong Kong Stock Exchange. The institutional environment in Hong Kong is much better than that of mainland China. They find that stocks of China-based firms have larger bid-ask spreads and thinner depth than stocks of H.K. based firms.

Better legal protection of investors encourages investing. During the early 2000s, property rights started to improve in China, but to varying degrees. \cite{RePEc:eee:jfinec:v:77:y:2005:i:1:p:117-146} consider two aspects of property rights: risk of expropriation by the government and contract enforcement's ease and reliability. Using data from a survey of firms based in 18 cities in China during 2000–2002, their study finds that when managers perceive the risk of expropriation to be low and the ease/reliability of enforcement to be high, they reinvest more of their firms' profit back into their firms.

Moreover, strong legal protection can encourage innovation. To test the role of intellectual property rights in China, \cite{RePEc:oup:rfinst:v:30:y:2017:i:7:p:2446-2477.} use survey data from 66 prefectures to examine the privatization of SOEs (state-owned enterprises). They find that on average, firms' patent stock increases by 200\% to 300\% in the five years after privatization compared to the five years before, and, more importantly, the increase in innovation is significantly more massive in prefectures with higher intellectual property rights protection. In 2007, China enacted the Property Right Law. \cite{RePEc:eee:jfinec:v:116:y:2015:i:3:p:583-593} study announcement returns on December 29, 2006, the day when a draft of the Property Law was accepted by the Standing Committee of the National People's Congress (NPC). The draft's acceptance was a surprise, as the NPC had debated this law for many years. On that day, the mean stock market return was almost 4\%. For event windows of (-2, +2) and (0, +5), where day 0 was December 29, 2006, the mean cumulative stock returns were over 6\% and over 15\%, respectively. These announcement returns are higher for firms with more tangible assets (that could have been expropriated by the local government) and without political connections (to prevent such expropriation).


\subsection{The Coasian View}

The emphasis on legal investor protection and market regulation stands in contrast with the traditional view that originated in Coase (1961) theorem. According to this perspective, most financial regulations are unnecessary because contracts take place between two sides. In other words, absent significant transaction costs, capital suppliers, and managers should negotiate, agree, and privately contract on the efficient level of investor protection when that level is not provided by law. Investors, on average, recognize the risk of expropriation and penalize firms that fail to disclose information. Because entrepreneurs bear these costs, they have the incentive to bind themselves contractually and limiting the expropriation. Hence as long as these contracts are enforced, financial markets do not require regulation.

However, this theorem crucially relies on courts enforcing elaborate contracts. Nevertheless, this kind of contract enforcement cannot be taken in most countries. Moreover, courts are often unwilling or unable to invest the resources to interpret complicated contracts, and they are slow, subject to political pressure, and sometimes corrupt. Therefore, in these circumstances, other forms of protecting property rights may be more efficient. It may be better to have contracts restricted by the law that are enforced than unrestricted contracts that are not.

The evidence suggests that private contracting is insufficient for corporate governance in general, especially for public firms. Consistent with this argument, \cite{RePEc:eee:jfinec:v:84:y:2007:i:3:p:738-771} find that Mexican private firms often offer significantly enhanced legal investor protection contract terms to their investors; meanwhile, public firms rarely do so. The Mexican law system does not protect financiers well enough, so private firms fill the gap left by the law by contractually opting out to the efficient level of investor protection, but why this behavior is not observed in the case of public firms? In practice, virtually no Mexican public firm offers significant investor protection to its public investors beyond that provided by the law. This contradicts the case for private companies and also at odds with the Coase theorem. To solve this puzzle, they construct a model that endogenizes the level of investor protection that firms provide, using the assumption that countries and different sets of legal laws differ in their ability to enforce elaborated contracts.

Their model shows, in an environment that contracts cannot be enforced precisely, public firms face a tradeoff in choosing the level of investor protection to offer. On the one hand, increasing the degree of investor protection has two upsides: first, it increases the raised capital and reduces the ex ante cost of underinvestment, and also it increases the firm’s income and decreases the extent of expropriation that is inefficient by nature. On the other hand, increasing investor protection generates contract overinclusion costs by suppressing the firm’s ability to take efficient actions at times. A private firm faces a similar tradeoff except that owing to its small number of investors, it can renegotiate contracts when it is necessary. Hence offering a high degree of investor protection and control rights to investors cannot prevent the firm from taking efficient actions by the possibility of contract renegotiation. Thus, in a legal regime with a limited degree of investor protection, private firms tend to offer contracts with enhanced legal protection more often than public firms. Due to their inability to renegotiate contracts, public firms are constrained to using contracts with the level of investor protection provided by the law, and so they are more sensitive to the legal system they operate in. This study also indicates a reliance of the contracting parties on the Mexican courts to enforce those contracts.

Finally, a critical implication of the legal investor protection approach to corporate governance is that leaving financial markets alone is not a good way to encourage them. They do need some protection of outsiders, whether by courts, government agencies, or private enforcing and market agents themselves. In sum, investor protection has broad consequences on ownership and control patterns, development of financial markets, the real part of the economy, institutional and legal reform, and resource allocation. Evidence shows that laws in some countries like the U.S., Japan, and Germany protect investors' rights relatively well; however, the legal system leaves managers with vast discretion in these countries. In most other countries, the laws are less protective, and law enforcement quality is worse comparatively; thus, legal protection alone is not enough for investors to get their money back.


\section{Large investors}

If the legal system is not sufficient for protecting investors’ rights, then perhaps investors can get more effective control rights by being large. Most governance mechanisms used in the world like relationship banking, large shareholding, institutional investors, takeovers, large creditors, can be viewed as large investors exercising their power. While large investors rely on the legal system, they do not require as many rights as the small investors do to protect their interests. Concentrated ownership, in effect, leverages up legal protection and can reduce agency costs. Nonetheless, large shareholding has its costs, which is discussed further in the following.

For presenting a big picture of ownership structure throughout the world, \cite{doi:10.1111/0022-1082.00115} document that except in countries with very good shareholder protection like the United States, relatively few firms are widely held. In fact, in most countries, virtually all firms have at least one controlling shareholder. Controlling shareholder is typically a family or the state and has control rights in excess of cash flow rights through pyramidal ownership structure or management participation. They show that for the sample of large firms, 36\% of the world's firms are widely held, 30\% are family-controlled, 18\% are State-controlled, and the remaining 15\% is divided between the residual categories. Also, Among the medium firms, the world average incidence of dispersed ownership is 24\%, compared to 36\% for the large firms; going down in size has the same effect as relaxing the strictness of the definition of control: it makes widely held firms scarcer. They note that it seems existing ownership structures are likely an equilibrium response to the domestic legal environments that companies operate in. Consistent with these findings, in another research, \cite{19455} argue that firms in poor investor protection countries may need to have concentrated ownership.

There is a free-rider problem inherent with small shareholders. Because small investors have small stakes in the firm, it does not pay them to monitor management and bear the monitoring cost. In contrast, large investors have a strong incentive to collect information and monitor management due to large stakes they own. Also, they have enough voting control over the firm to put pressure on management. Because they have a large ownership stake, it pays to them to maximize firm valuation and therefore minimizing expropriation and agency costs.

Although concentrated ownership has its benefits, it also has costs as well. The cost of having a large shareholder is best described by potential expropriation from minority shareholders by the controlling shareholder. This is the central problem of corporate governance in most of the world. For example, in China, the main agency problem is horizontal agency conflict between controlling and minority shareholders arising from concentrated ownership structure. Controlling shareholders may treat themselves preferentially at the expense of other investors and employees. They may pay themselves special dividends by using business relations with other firms they control, like the case often happens in business groups. The fact that shares with superior control rights trade at significant premiums is evidence of considerable private benefits of control.

Typically, ownership and control concentration has two effects on firm value. The first is more concentrated cash flow rights in the hands of controlling shareholder; the stronger is his incentive to run the firm properly because when the firm is run properly, his wealth is maximized. This effect is in a positive relationship between the firm value and cash flow rights of controlling shareholder. The second effect is the entrenchment effect. In contrast with the first effect, the more control rights are in the hands of controlling shareholder, the more entrenched he is and thus better able to extract the firm’s wealth at the expense of minority shareholders. This effect suggests a negative relationship between control rights of controlling shareholder and the firm value. The negative effect of entrenchment is more severe when there is a divergence between cash flow rights and control rights that often occurs through pyramids because the willingness to extract value is less restrained by the cash-flow stake.

A straightforward inference of these arguments is that when controlling shareholder's cash-flow ownership is higher, lower expropriation by controlling shareholder and more firm valuation is expected. \cite{doi:10.1111/1540-6261.00457} present a model of the relationship between legal protection of minority shareholders and cash-flow ownership by the controlling shareholder on the firm valuation. This model is presented briefly in the following.


\subsection{A model of controlling shareholder}

In this model, there is a firm fully controlled by a single controlling shareholder. This controlling shareholder has a cash flow stake in the firm, which is assumed exogenous, and there is no sale of equity by the controlling shareholder. As a private benefit of control, the controlling shareholder can divert a fraction of the profits before distributing the rest as dividends. This diversion has costs. These costs are the share of profits he wastes during tunneling and are a function of the share of profits he decided to divert and also the quality of investor protection of the country the firm operates in. Stealing is costlier in a more protective legal regime, and the marginal cost of stealing is positive. The model results that the higher is the cash-flow ownership by the controlling shareholder, the greater are his incentives to distribute dividends in a non-distortionary way rather than expropriate minority shareholders in a distortionary way, and hence the lower is the equilibrium level of expropriation for a fixed legal protection level. Simply high cash-flow ownership by the controlling shareholder reduces minority expropriation. Moreover, It can be shown that by the assumptions of this model, in countries with better shareholder protection, there is less expropriation of minority shareholders.

\subsection{Evidence of the Effects of Large Investors}

In a celebrated study, \cite{doi:10.1111/1540-6261.00511}, using data for 1,301 publicly traded corporations in eight East Asian economies, find that firm valuation increases with the largest shareholder's cash-flow ownership. More specifically, they show that concentrated ownership in the hands of all types of owners is associated with a higher market to book ratio. Moreover, they show that firm value falls when controlling shareholders' control rights exceed its cash flow ownership. These results are consistent with the positive incentive effect associated with increased cash-flow rights of controlling shareholder and the negative entrenchment effect with the sizeable controlling shareholder. They document that this negative effect is particularly severe when the deviation between controlling and ownership rights is higher. 

They find that East Asian firms also show a sharp divergence between cash-flow rights and control rights. That is, the largest shareholder is often able to control a firm's operations with a relatively small direct stake in its cash-flow rights. Control is often enhanced beyond ownership stakes through pyramid structures and cross-holdings among firms, and sometimes through dual-class shares. They document the divergence between cash-flow rights and control rights is most pronounced in family-controlled firms. Their main contribution to the literature in this study is disentangling the incentive and entrenchment effect of large ownership that are so difficult to tell apart in the U.S. market because, in the U.S., firms are typically widely held.

In an earlier study, \cite{CLAESSENS200081} used data for 2,980 corporations in nine East Asian countries. Find that in all countries, corporate control is typically enhanced by pyramid structures and cross-holdings. This study shows that the separation of ownership and control is most pronounced among family-controlled firms and small firms. They also find that controls more than two-thirds of listed firms have a single controlling shareholder. Separation of management from ownership control is rare, and management of 60\% of the not widely held firms is related to the family of the controlling shareholder; Older firms are more likely family-controlled, which dispels the notion that dispersion of ownership is just a matter of time. Finally, significant corporate wealth in East Asia is concentrated among a few families.

Consistent with the negative entrenchment effect of the controlling shareholder, a study on Korean business groups by \cite{doi:10.1111/1540-6261.00510} reveals that tunneling often occurs in business groups by the controlling shareholder. They find that when a group-affiliated firm makes an acquisition, its stock price on average falls. While minority shareholders of a firm making an acquisition lose, the controlling shareholder of that firm on average benefits because the acquisition enhances other firms' value in the group. In a similar study, \cite{RePEc:oup:qjecon:v:117:y:2002:i:1:p:121-148.} show that in Indian business groups, a significant amount of tunneling happens by the controlling shareholder, much of it occurs via nonoperating components of profit. Their interesting methodology rests on isolation and testing distinctive implications of the tunneling hypothesis for the propagation of earning shocks across firms within a group.

Family ownership of firms is common around the world as a large shareholder. In a study, \cite{doi:10.1111/1540-6261.00567} find that family ownership is prevalent and substantial among the S\&P 500 firms and account for 18\% of the outstanding equity. Contrary to their conjecture, they find that family firms perform better on average than non-family firms, and the performance is even better when family members serve as CEO than with outside CEOs. Based on the average ROA in the sample, family firms appear to return 6.65 percent more than non-family firms; this is economically huge. Their results suggest that family ownership is an effective organizational structure.

\subsection{Institutional investors}

Institutional investors are instances of large investors with significant effects. Here the role of institutional investors in corporate governance is discussed briefly by explaining the results of some empirical studies. Before them, it is worth mentioning that institutional ownership of listed firms has increased substantially over recent decades. According to the Federal Reserve Board's Flow of Funds report, institutions owned approximately 7\% of U.S. equities in 1950 and 51\% by the end of 2004. According to the International Monetary Fund (2005), these professional investors manage financial assets exceeding 45 trillion USD (including over US\$20 trillion in equities). So, understanding their role in financial markets is an important issue. Institutional investors are major players not just in developed countries but also in emerging market countries.

Institutional involvement in corporate governance can have various forms, such as threatening the sale of shares, actual share sale known as the "wall street walk", actively using voting rights, and meeting and influencing management. The question is whether these large types of investors are useful in monitoring and influencing management towards creating shareholder value. It is reasonable to think that not all money managers are equally equipped or motivated to be active monitors. For instance, \cite{RePEc:eee:jfinec:v:68:y:2003:i:1:p:3-46} conclude that rather than exerting effort to influence management, some institutional investors vote with their feet by selling their shares (wall street walk) when they are dissatisfied with corporate performance.

The evidence for the debate on whether institutional investors' monitoring and activism is effective is somewhat mixed. \cite{RePEc:eee:jfinec:v:68:y:2003:i:1:p:3-46} find that institutional selling influences the decision of the board of directors to fire a CEO, while Gillan and Starks (2003) find typically modest stock price reactions to shareholder proposals by activist institutions. McConnell and Servaes (1990) detect a positive relation between Tobin's Q and the fraction of shares owned by institutions. However, Woidtke (2002) shows that firm value is positively related only to ownership by private pension funds, not the other types of institutional investors.

Institutional monitoring consists of two elements: first, information gathering, which institutions use it for portfolio selection decisions, and second, using such information to actively influencing firm policies and thereby benefiting all investors. It is highly plausible that some institutions prefer one aspect over another.

\cite{RePEc:eee:jfinec:v:88:y:2008:i:3:p:499-533} study institutional investors' role around the world using a comprehensive dataset. They find all institutional investors seek large firms and firms with strong governance indicators, but foreigners overweight firms in the Morgan Stanley Capital International World Index and firms cross-listed on a U.S. exchange. Moreover, their findings suggest that some (but not all) institution groups are effective monitors of the firms they invest in. They also find that the presence of foreign and independent institutions enhances shareholder value. Unlike domestic institutions, these institutions can exert pressure because they have fewer business relations with the firm to jeopardize. Their tests show that firms held by foreign and independent institutions have higher valuations; there is no similar evidence for ownership by domestic institutions. Furthermore, they document that foreign and independent institutions are associated with better operating performance and reduced capital expenditures. Their results are robust in many several ways.

In another research considering the monitoring role of institutions, \cite{RePEc:eee:jfinec:v:86:y:2007:i:2:p:279-305} find that independent long-term institutions (ILTIs) monitor firms effectively, and other types of institutional investors will not monitor. Their study uses a cost-benefit framework and hypothesizes that ILTIs will invest in monitoring and influencing firms rather than trading. Using acquisition data, they show that the presence of ILTIs is related to better post-merger performance, and also the likelihood of withdrawal of bad bids is higher. They document that these institutions make long-term portfolio adjustments rather than trading for short-term gain. In sum, for ILTIs, the gain from effective monitoring is shared with other shareholders, while the gain from long-run portfolio adjustment is private.


\section{Financial analysts as an external corporate governance mechanism}

In previous sections, the role of legal protection and large investors have been discussed. However, this is not the entire story. There are more known corporate governance mechanisms and probably numerous unknowns. In this section, as of another form of corporate governance, we briefly discuss about financial analysts.

Analysts can serve as an external governance mechanism through at least two channels. First, analysts track firms' financial statements regularly and directly interface with management by raising questions in earnings announcement conference calls, which can be regarded as direct monitoring. Second, analysts provide indirect monitoring by distributing public and private information to institutional investors and millions of individual investors through research reports and media, helping investors detect managerial misbehavior.

\cite{RePEc:eee:jfinec:v:115:y:2015:i:2:p:383-410} examine analysts' role as an external corporate governance mechanism in mitigating agency conflicts over corporate policies. Their study relies on two natural experiments which are exogenous. They scrutinize the causal effects of analysts' coverage on managerial expropriation of outside shareholders. They find that as a firm experiences an exogenous decrease in analyst coverage, shareholders value internal cash holdings less. Its CEO receives higher excess compensation. Its management is more likely to make value-destroying acquisitions, and its managers are more likely to engage in earnings management activities. Also, they find that these effects are more pronounced in two subsamples of firms first for firms with lower initial analyst coverage where losing an analyst is more important marginally; second, for those with less product market competition, meaning there are fewer substitutes constraints on agency problems. These findings show that financial analysts play an important governance role.






  



%% The Appendices part is started with the command \appendix;
%% appendix sections are then done as normal sections
%% \appendix

%% \section{}
%% \label{}

%% References
%%
%% Following citation commands can be used in the body text:
%% Usage of \cite is as follows:
%%   \cite{key}          ==>>  [#]
%%   \cite[chap. 2]{key} ==>>  [#, chap. 2]
%%   \citet{key}         ==>>  Author [#]

%% References with bibTeX database:

\bibliographystyle{chicago}
\bibliography{BiblographyDatabase.bib}
%% Authors are advised to submit their bibtex database files. They are
%% requested to list a bibtex style file in the manuscript if they do
%% not want to use model1-num-names.bst.

%% References without bibTeX database:

% \begin{thebibliography}{00}

%% \bibitem must have the following form:
%%   \bibitem{key}...
%%

% \bibitem{}

% \end{thebibliography}


\end{document}
